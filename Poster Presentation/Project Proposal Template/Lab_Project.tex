\documentclass[12pt]{report}
\usepackage[a4paper, left=3.17cm, right=3.17cm, top=2.54cm, bottom=2.54cm]{geometry}
\usepackage[T1]{fontenc}
\usepackage{mathptmx}
\usepackage{amsmath}
\usepackage{amsfonts}
\usepackage{chemformula}
\usepackage{multicol}
\usepackage{multirow}
\usepackage{tabularx,booktabs}
\newcolumntype{C}{>{\centering\arraybackslash}X} % centered version of "X" type
\usepackage[linesnumbered,ruled,vlined]{algorithm2e}
\usepackage{comment}
\usepackage{array}
\newcolumntype{P}[1]{>{\centering\arraybackslash}p{#1}}
\usepackage{cite}
\usepackage[colorlinks, linkcolor=black, anchorcolor=black, citecolor=black]{hyperref}
\usepackage{graphicx}
\setlength{\parskip}{0.5em}
\title{Place Your Project Title at Here}
%\author{\textup{Qi YUAN}}


%copyright at footer
\usepackage{fancyhdr}
\fancyhf{}
\rfoot{%
  \footnotesize
  \textcopyright~Dept. of Computer Science and Engineering, GUB\\

 }
%\pagestyle{fancy}


\begin{document}
    \begin{titlepage}
\center 
\newcommand{\HRule}{\rule{\linewidth}{0.1mm}}
\includegraphics[scale=0.6]{Figures/GUB.png}\\[1cm] 
\center 
%\quad\\[1.5cm]
\textsl{\Large Green University of Bangladesh }\\[0.5cm] 
\textsl{\large Department of Computer Science and Engineering (CSE)}\\
%\textsl{\large Faculty of Sciences and Engineering (FSE)}\\
\textsl{\large Semester: (Fall, Year: 2024), B.Sc. in CSE (Day)}\\[0.5cm] 
\makeatletter
\HRule \\[0.2cm]
{ \Large \bfseries \@title}\\[0.1cm] 
\HRule \\[1.0cm]

\textsl{\large Course Title: CSE }\\
\textsl{\large Course Code: CSE-000 }\\ 
\textsl{\large Section: 000-00 }\\[0.5cm] 

{\large \underline{Students Details}}\\[0.2cm]

\begin{table*}[htb]
\centering
\begin{tabular}{ |P{7.0cm}|P{3.5cm}|}
\hline
\textbf{Name} & \textbf{ID}\\
\hline
A   & 1 \\
\hline
B & 2 \\
\hline
C & 3 \\
\hline
\end{tabular}
\end{table*}
\vspace{0.5cm}


\textsl{\large Submission Date: \ 00.00.0000 }\\ 
\textsl{\large Course Teacher’s Name: \ Md. Romzan Alom }\\[0.9cm] 




\makeatother
{\large [For teachers use only: \textcolor{red}{Don’t write anything inside this box}]}\\

\begin{table}[]
\centering
\begin{tabular}{|p{7.5cm}p{7.0cm}|}
\hline
\multicolumn{2}{|c|}{{\underline{\textbf{Project Proposal Status}}}} \\
 & \\\hline
\textbf{Marks:}                & \textbf{Signature:}        \\
 & \\ 
\textbf{Comments:}             & \textbf{Date:}             \\ 
 & \\\hline
\end{tabular}
\end{table}


\end{titlepage}


    \tableofcontents
  


% Chapter 1 starting here.....    
\newpage
\chapter{Introduction}

\section{Overview}

\section{Problem Domain}
\section{Motivation}
\section{Objective}


\chapter{Literature Review}

\chapter{Methodology}
The block diagram visually organizes these steps into a logical sequence. It uses clearly labeled blocks connected by directional arrows, signifying the flow of work.
\chapter{Feasibility Study}

\section{Technical Feasibility}
\section{Operational Feasibility}
\section{Financial Feasibility}
\section{Project Management}

\chapter{Social Impact and Benefit}
\chapter{Conclusion}

% References starts here... 
\newpage
\renewcommand\bibname{References}
\chapter{References}
\bibliographystyle{unsrt}
\bibliography{Ref}

\end{document}